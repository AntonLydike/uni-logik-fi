\documentclass[leqno]{article}
\usepackage{microtype}
\usepackage[utf8]{inputenc}
\usepackage[a4paper, total={7in, 9.6in}]{geometry}
\usepackage{enumerate}
\usepackage{amsmath}
\usepackage{amssymb}
\usepackage{mathabx}
\usepackage{fancyhdr}
\usepackage{xcolor}
\usepackage{mathtools}
\usepackage{graphicx}
\usepackage{svg}


%% headers
\pagestyle{fancy}
\fancyhf{}
\rhead{Logik Für Informatiker WS19/20}
\lhead{Daniel Schubert, Anton Lydike}
\rfoot{Seite \thepage}

\usepackage{etoolbox}
\AtBeginEnvironment{align}{\setcounter{equation}{0}}

% simple command to display Aufgabe <num>)       ___ / <num>p.
\newcommand\task[2]{\section*{Aufgabe #1)\hfill \underline{\,\,\,\,\,\,}\,\,/#2p.}}

% Interpretation (I)
\newcommand\I{I}
% Interpretation und belegung (I, \beta)
\newcommand\Ib{\I, \beta}
% Teilinterpretation (J)
\newcommand\J{J}
% Teilinterpretation und belegung (J, \beta)
\newcommand\Jb{\J, \beta}

%% models
%\newcommand{\models}{\rightmodels}
\renewcommand{\models}{\vDash}
\newcommand\lmodels{\Dashv} 			% =|
\newcommand\bimodels{\lmodels\models}	% =||=


\newcommand\zuzeigen{\mathrm{Z\kern-.3em\raise-0.5ex\hbox{Z}}}

%% table for total points
\newcommand\pointsttl[1]{\section*{Gesamtpunkte: \hfill \underline{\,\,\,\,\,\,}\,\,/#1p.}}

%% Funktionen und Prädikate
% Funktionen (arg ist anzahl der stellen)
\newcommand\func[1]{\mathcal{F}^{#1}}
% Prädikate (arg ist anzahl der stellen)
\newcommand\praed[1]{\mathcal{P}^{#1}}

%% Regeln
\newcommand\defrule[2]{\frac{#1}{#2}}

%% Funktionszahl
\newcommand\funcnum[1]{\#_{F}\, #1}

% Für ersetzungen in belegungen wie { x \mapsto d }
\newcommand\repl[2]{\{#1 \mapsto #2\}}

% für alle x .
\newcommand\fall[1]{\forall #1 \, . \,}
\newcommand\ex[1]{\exists #1 \, . \,}

% short biimplication
\newcommand\biimpl{\Leftrightarrow}

% draw a box on the right side of the page
\newcommand\qed{ \hfill $\Box$ }

% red, green, blue text:
\newcommand\red[1]{\textcolor{red}{#1}}
\newcommand\green[1]{\textcolor{green}{#1}}
\newcommand\blue[1]{\textcolor{blue}{#1}}

% more symbols: https://oeis.org/wiki/List_of_LaTeX_mathematical_symbols

\newcommand\cfgtitle[1]{\title{\vspace{-1.5cm}Übungsblatt #1\\%
\begin{large} Übungsgruppe 1 \end{large}} \lfoot{Übungsblatt #1}\cfoot{Übungsgruppe 1}}
\author{Daniel Schubert\\Anton Lydike}

\newcommand\smiley{\vfill\centering\includesvg[scale=.5]{../../smile.svg}}


\newcommand\N{\mathbb{N}}
\newcommand\Z{\mathbb{Z}}

\newcommand\ff{f\!\! f}


\usepackage{enumitem}
\usepackage{multirow,tabularx}
\usepackage{wasysym}

\newlist{axioms}{enumerate}{1}
\setlist[axioms]{label=Ax\arabic*)}



% Logik f.I.

\cfgtitle{6}
\date{Donnerstag 21.11.2019}

\newcommand\X{\mathcal{X}}


\begin{document}
\maketitle
\thispagestyle{fancy}

\task{1}{10}

\begin{axioms}

\item \begin{align}
	 \{A,B\} & \vdash A & \text{(A)} \\
	 \{A,B\} & \vdash B & \text{(A)} \\
	 A & \vdash B \to A & \text{(\to R)} \\
	 & \vdash A\to (B\to A) & \text{(\to R)}
\end{align}

\item \begin{align}
	 \{A,\lnot C, B\} & \vdash \lnot C & \text{(A)} \\
	 \{A,\lnot C, B\} &\vdash A & \text{(A)} \\
	 \{A,\lnot C, B\} &\vdash B & \text{(A)} \\
	 \{A \to B,\lnot C, A\} &\vdash B & \text{(\to L)} \\
	 \{A \to B, C, A\} &\vdash C & \text{(A)} \\
	 \{A \to B, \lnot C, A\} &\vdash A & \text{(A)} \\
	 \{A \to B, A, B\to C\} &\vdash C & \text{(\to L)} \\
	 \{A \to (B\to C), A \to B, A\}& \vdash C & \text{(\to L)} \\
	 \{A \to (B\to C), A \to B\} &\vdash A\to C & \text{(\to R)} \\
	 \{A \to (B\to C) \} &\vdash (A \to B) \to (A\to C) & \text{(\to R)} \\
	& \vdash (A \to (B\to C)) \to ((A \to B) \to (A\to C)) & \text{(\to R)}
\end{align}


\item \begin{align}
	 \{B, \lnot A\} & \vdash B & \text{(A)} \\
	 \{B, \lnot A\} & \vdash \lnot A & \text{(A)} \\
	 \{B, \lnot B\} & \vdash A & \text{(\lnot R)} \\
	 \{B, \lnot A \to \lnot B\} & \vdash A & \text{(\to L)} \\
	 \{\lnot A \to \lnot B\} & \vdash B \to A & \text{(\to R)} \\
	  & \vdash (\lnot A \to \lnot B) \to (B \to A)& \text{(\to R)}
\end{align}

\end{axioms}


\task{2}{6}

Sei $(M_i)_{i \in \N}$ eine beliebige folge endlicher, konsistenter Mengen mit $M_i \subset M_{i+1} \forall i\in \N$. Es ist zu zeigen, dass $ M := \bigcup_{i=1}^{\infty} M_i$ wieder konsistent ist.

\bigskip\noindent Angenommen $M$ wäre inkonsistent. Dann müsste $M$ $A$ und $\lnot A$ enthalten. Wir definieren nun $M_0 := \emptyset$ und $\hat{M}_i := M_{i} \setminus M_{i-1}$ als die (endliche) Menge an Aussagen, die im $i$-ten Schritt hinzukommen. Da $\{A, \lnot A\} \subset M$ und alle $M_i$ endlich sind, finden wir $j > 2$ mit:

\pagebreak

\begin{enumerate}
\item $A \in \hat{M}_j$ und $\lnot A \in M_{j-1}$ oder 
\item $\lnot A \in \hat{M}_j$  und $A \in M_{j-1}$ oder 
\item $\{A, \lnot A\} \subset \hat{M}_j$
\end{enumerate}

\bigskip\noindent In jedem Fall ist einfach zu sehen, dass $M_{j-1}$ konsistent, aber $M_j$ inkonsistent ist, da aus $M_{j-1} \subset M_j$ folgt, dass $\{A, \lnot A\} \subset M_j$ gilt, was ein Wiederspruch zur Angabe ist, dass alle $M_i$ konsistent sind.

\task{3}{9}

Zu zeigen, $M \cup \{A \land B\} \bimodels M \cup \{A,B\}$:

\begin{itemize}
	\item[,,$\models$''] Sei $\Ib$ beliebig und gelte $\Ib \models M \cup \{A\land B\}$. Dann folgt, $\Ib \models M$ und $\Ib \models A\land B$. Insbesondere also $\Ib \models A$ und $\Ib \models B$. Damit folgt $\Ib \models M \cup \{A,B\}$
	\item[,,$\lmodels$''] Sei $\Ib$ beliebig und gelte $\Ib \models M \cup \{A, B\}$. Dann folgt $\Ib \models A$, $\Ib \models B$ und $\Ib \models M$. Daraus folgt, $\Ib \models A\land B$. Damit folgt, $\Ib \models M \cup \{A \land B\}$. \qed
\end{itemize}

\begin{enumerate}

\item \begin{align}
	& \{\lnot q, r \to (\lnot p \to q), \lnot (p \lor \lnot r)\} \vdash \\	
	& \{\lnot q, r \to (\lnot p \to q), \lnot p \land r)\} \vdash & \text{(De Morgan)} \\
	& \{\lnot q, r \to (\lnot p \to q),\lnot p, r\} \vdash & \text{(Hinweis)} \\
	& \{\lnot q, r \to (\lnot p \to q),\lnot p, r\} \vdash r \to (\lnot p \to q)) & \text{(Trivial)} \\
	& \{\lnot q, r \to (\lnot p \to q),\lnot p, r\} \vdash r & \text{(Trivial)} \\
	& \{\lnot q, r \to (\lnot p \to q),\lnot p, r\} \vdash \lnot p \to q & \text{(MP(3)(4)} \\
	& \{\lnot q, r \to (\lnot p \to q),\lnot p, r\} \vdash \lnot p & \text{(Trivial)} \\
	& \{\lnot q, r \to (\lnot p \to q),\lnot p, r\} \vdash q & \text{(MP(6)(7)} \\
	& \{\lnot q, r \to (\lnot p \to q),\lnot p, r\} \vdash \lnot q \text{ \lightning}  & \text{(Trivial)} 
\end{align}

\item \begin{align}
	& \{\lnot p \lor (q \to r),  p \to q, \lnot (p \to r)\} \vdash \\
	& \{p \to (q \to r),  p \to q, \lnot (p \to r)\} \vdash & \text{Meta}\\
	& \{p \to (q \to r),  p \to q, \lnot (p \to r)\} \vdash p \to (q \to r) & \text{Trivial}\\
	& \{p \to (q \to r),  p \to q, \lnot (p \to r)\} \vdash (p \to (q \to r)) \to ((p \to q) \to (p \to r)) & \text{Ax2}\\
	& \{p \to (q \to r),  p \to q, \lnot (p \to r)\} \vdash (p \to q) \to (p \to r) & \text{MP(3)(4)}\\
	& \{p \to (q \to r),  p \to q, \lnot (p \to r)\} \vdash (p \to q) & \text{Trivial}\\
	& \{p \to (q \to r),  p \to q, \lnot (p \to r)\} \vdash (p \to r) & \text{MP(5)(6)}\\
	& \{p \to (q \to r),  p \to q, \lnot (p \to r)\} \vdash \lnot (p \to r) \text{ \lightning}& \text{Trivial}
\end{align}

\item $$ M_3 := \{\lnot p \leftrightarrow q, p \to r, \lnot p\} $$

Wenn $p^I = \ff$, $q^I=tt$ und $r=tt$, dann ist $M_3$ erfüllbar, und damit laut Skript (Satz 2.5) auch konsistent.

\end{enumerate}

\pointsttl{25}

\vfill\centering\includesvg[scale=.5]{../../smile.svg}
\end{document}