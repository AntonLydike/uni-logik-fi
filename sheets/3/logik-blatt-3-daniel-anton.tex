\documentclass{article}
\usepackage{microtype}
\usepackage[utf8]{inputenc} 
\usepackage[a4paper, total={6in, 9.6in}]{geometry}
\usepackage{MnSymbol}
\usepackage{stmaryrd}
\usepackage{enumerate}
\usepackage{amsmath}
\usepackage{fancyhdr}
\usepackage{xcolor}
\usepackage{mathtools}

%% headers
\pagestyle{fancy}
\fancyhf{}
\rhead{Logik Für Informatiker WS19/20}
\lhead{Daniel Schubert, Anton Lydike}
\rfoot{Seite \thepage}

% simple command to display Aufgabe <num>)       ___ / <num>p.
\newcommand\task[2]{\section*{Aufgabe #1)\hfill \underline{\,\,\,\,\,\,}\,\,/#2p.}}

% Interpretation (I)
\newcommand\I{I}
% Interpretation und belegung (I, \beta)
\newcommand\Ib{\I, \beta}

%% models
\newcommand\lmodels{\leftmodels} 			% =|
\newcommand\bimodels{\leftmodels\models}	% =||=


%% table for total points
\newcommand\pointsttl[1]{\section*{Gesamtpunkte: \hfill \underline{\,\,\,\,\,\,}\,\,/#1p.}}

%% Funktionen und Prädikate
% Funktionen (arg ist anzahl der stellen)
\newcommand\func[1]{\mathcal{F}^{#1}}
% Prädikate (arg ist anzahl der stellen)
\newcommand\praed[1]{\mathcal{P}^{#1}}

%% Regeln
\newcommand\defrule[2]{\frac{#1}{#2}}

%% Funktionszahl
\newcommand\funcnum[1]{\#_{F}\, #1}

% Für ersetzungen in belegungen wie { x \mapsto d }
\newcommand\repl[2]{\{#1 \mapsto #2\}}

% für alle x .
\newcommand\fall[1]{\forall #1 \, . \,}
\newcommand\ex[1]{\exists #1 \, . \,}

% short biimplication
\newcommand\biimpl{\Leftrightarrow}

% draw a box on the right side of the page
\newcommand\qed{ \hfill $\Box$ }

% red, green, blue text:
\newcommand\red[1]{\textcolor{red}{#1}}
\newcommand\green[1]{\textcolor{green}{#1}}
\newcommand\blue[1]{\textcolor{blue}{#1}}

% more symbols: https://oeis.org/wiki/List_of_LaTeX_mathematical_symbols

\newcommand\cfgtitle[1]{\title{\vspace{-1.5cm}Übungsblatt #1\\%
\begin{large} Übungsgruppe 1 \end{large}} \lfoot{Übungsblatt #1}\cfoot{Übungsgruppe 1}}
\author{Daniel Schubert\\Anton Lydike}
% Logik f.I.

\cfgtitle{3}
\date{Donnerstag 07.11.2019}

\begin{document}
\maketitle
\thispagestyle{fancy}

\task{1}{6}

\begin{enumerate}
	\item $\I \models A \to B \biimpl \I \not\models A \text{ oder } \I \models B$

	\begin{enumerate}
		
		\item[,,$\Rightarrow$"]
		
		Betrachte:		
		
		\begin{align*}
				A &= \text{is\_divisble\_by\_four}(x) \\
				B &= \text{is\_even}(x) \\ 
		\end{align*}
		
		Daraus folgt, dass $A\to B$ für alle $\beta$ wahr ist. Jedoch gilt weder
		
		\begin{align*}
				\I \not\models & \text{ is\_divisble\_by\_four}(x) \\
				\I \models & \text{ is\_even}(x) \\ 
		\end{align*}
		
		für alle belegungen $\beta$. So gilt z.B. 
		
		\begin{align*}
				\Ib \models A & & \text{ für } \beta & = \{4,8\} 	 \\
				\I,\hat\beta \not\models B & & \text{ für } \hat\beta & =\{1,2,3,4\} 
		\end{align*}
		
		
		Somit gilt weder $\I\not\models A$ noch $\I\models B$ für alle $\beta$. \qed
		
		
		\item[,,$\Leftarrow$"]
		
		\begin{align*}
			\I \not\models A \text{ oder } \I \models B & \Rightarrow \text{für alle } \beta \text{ gilt } \Ib \not\models A \text{ oder für alle } \hat{\beta} \text{ gilt } \I, \hat{\beta} \models B	& (\text{Def. Modell})\\
			& \Rightarrow \text{für alle } \beta \text{ gilt } (\Ib \not\models A \text{ oder } \Ib \models B)  					& (\text{Insb. $\beta = \hat\beta$})\\
			& \Rightarrow \text{für alle } \beta \text{ gilt } (\Ib \models \lnot A \text{ oder } \Ib \models B ) 	& (\text{A3})\\
			& \Rightarrow \text{für alle } \beta \text{ gilt } \Ib \models \lnot A \lor B 							& (\text{A4})\\
			& \Rightarrow \text{für alle } \beta \text{ gilt } \Ib \models \lnot A \lor \lnot(\lnot B) 			& (\text{Meta})\\
			& \Rightarrow \text{für alle } \beta \text{ gilt } \Ib \models A \to B									& (\text{De Morgan})\\
			& \Rightarrow \I \models A \to B													& (\text{Def. Modell})\\
		\end{align*}		
		
		\qed		
		
	\end{enumerate}
	
	\item $\Ib \models \fall{x} A \Rightarrow \Ib \models\ex{x} A $
	
		\begin{align*}
			\Ib \models \fall{x} A & \Rightarrow \text{ für alle $d\in D$ gilt } \Ib \repl{x}{d} \models A & (\text{A5})\\
			& \Rightarrow \text{ es existiert ein $d\in D$ mit } \Ib \repl{x}{d} \models A & (\text{Meta})\\
			& \Rightarrow \Ib \models\ex{x} A & (\text{A5})
		\end{align*}
	
	
	\qed
\end{enumerate}

\task{2}{9}

\begin{enumerate}
	\item 
	
	\begin{enumerate}
		\item \textbf{Ja}, da die Klammerung um die linke Seite komplett gültig ist
		\item \textbf{Nein}, da $A \to B$ nicht aus Regeln hergeleitet werden kann (siehe 4.)
		\item \textbf{Nein}, analog zu (b)
		\item \textbf{Nein}, da $\exists$-Quantor nicht für universelle Formeln zugelassen ist (siehe 3.)
		\item \textbf{Nein}, da weder die implikation, noch die biimplikation zulässig sind
	\end{enumerate}	

	\item Betrachte $(\fall{x} P(x))\land \fall{y}Q(y)$ mit Interpretationen $I, J$ und $I \subset J$:
	
	\begin{center}
		\begin{tabular}{l l l}
			$D_J := \{\triangle, \Box, \circ, \medstar, \between\}$ & $P^J(x) := x$ ist konvex & $Q^J(x) := tt$ \\
			$D_I := \{\triangle, \Box\} \subset D_J$ 		   & $P^I(x) := x$ hat Ecken & $Q^I(x) := tt$\\
		\end{tabular}
	\end{center}
	
	Es folgt, dass $\forall d \in D_I : P^I(d) \iff P^J(d)$ und $\forall d \in D_I : Q^I(d) \iff Q^J(d)$, jedoch ist $P^I(\medstar) \not\iff P^J(\medstar)$, weshalb zwar $\I\models (\fall{x} P(x))\land \fall{y}Q(y)$ gilt, $J \models (\fall{x} P(x))\land \fall{y}Q(y)$ jedoch nicht.
	
	\bigskip Es st nur für die erste Formel möglich, da nur die erste Formel universell ist.
	
	\item Es existiert keine Regel, die es erlaubt den $\exists$-Quantor herzuleiten.

	\item \textit{*Intuitive Begründung, weshalb Pfeile böse sind*}

\end{enumerate}

\task{3}{10}


\begin{enumerate}
	\item $\fall{y} (P(y) \land P(x)) \to LE(x,y)  $
	\item $\ex{x} P(x) \land \fall{y} LE(x,y) $
	\item $\ex{x} \fall{y} x+y=y$
	\item $\ex{x} P(x) \land \fall{y} P(y) \to LE(y,x)$
	\item Ja gibt es. Mit $D=\{1,2,3,4,5,6,7,8\}$ folgt, dass 7 die größte Primzahl ist.
	\item $(\ex{x,y} x \not = y \land P(x) \land P(y)) \to \lnot A$
\end{enumerate}

\pointsttl{25}
\smiley

\end{document}