\documentclass{article}
\usepackage{microtype}
\usepackage[utf8]{inputenc} 
\usepackage[a4paper, total={6in, 9.6in}]{geometry}
\usepackage{MnSymbol}
\usepackage{stmaryrd}
\usepackage{enumerate}
\usepackage{amsmath}
\usepackage{fancyhdr}
\usepackage{xcolor}
\usepackage{mathtools}

%% headers
\pagestyle{fancy}
\fancyhf{}
\rhead{Logik Für Informatiker WS19/20}
\lhead{Daniel Schubert, Anton Lydike}
\rfoot{Seite \thepage}

% simple command to display Aufgabe <num>)       ___ / <num>p.
\newcommand\task[2]{\section*{Aufgabe #1)\hfill \underline{\,\,\,\,\,\,}\,\,/#2p.}}

% Interpretation (I)
\newcommand\I{I}
% Interpretation und belegung (I, \beta)
\newcommand\Ib{\I, \beta}

%% models
\newcommand\lmodels{\leftmodels} 			% =|
\newcommand\bimodels{\leftmodels\models}	% =||=


%% table for total points
\newcommand\pointsttl[1]{\section*{Gesamtpunkte: \hfill \underline{\,\,\,\,\,\,}\,\,/#1p.}}

%% Funktionen und Prädikate
% Funktionen (arg ist anzahl der stellen)
\newcommand\func[1]{\mathcal{F}^{#1}}
% Prädikate (arg ist anzahl der stellen)
\newcommand\praed[1]{\mathcal{P}^{#1}}

%% Regeln
\newcommand\defrule[2]{\frac{#1}{#2}}

%% Funktionszahl
\newcommand\funcnum[1]{\#_{F}\, #1}

% Für ersetzungen in belegungen wie { x \mapsto d }
\newcommand\repl[2]{\{#1 \mapsto #2\}}

% für alle x .
\newcommand\fall[1]{\forall #1 \, . \,}
\newcommand\ex[1]{\exists #1 \, . \,}

% short biimplication
\newcommand\biimpl{\Leftrightarrow}

% draw a box on the right side of the page
\newcommand\qed{ \hfill $\Box$ }

% red, green, blue text:
\newcommand\red[1]{\textcolor{red}{#1}}
\newcommand\green[1]{\textcolor{green}{#1}}
\newcommand\blue[1]{\textcolor{blue}{#1}}

% more symbols: https://oeis.org/wiki/List_of_LaTeX_mathematical_symbols

\newcommand\cfgtitle[1]{\title{\vspace{-1.5cm}Übungsblatt #1\\%
\begin{large} Übungsgruppe 1 \end{large}} \lfoot{Übungsblatt #1}\cfoot{Übungsgruppe 1}}
\author{Daniel Schubert\\Anton Lydike}


% Logik f.I.

\cfgtitle{4}
\date{Donnerstag 14.11.2019}

\newcommand\X{\mathcal{X}}

\newcommand{\DNF}[1]{\mathrm{\,DNF({#1})}}
\newcommand{\KNF}[1]{\mathrm{\,KNF({#1})}}

\begin{document}
\maketitle
\thispagestyle{fancy}

\task{1}{5}

\begin{itemize}
	\item[] Sei $I$ eine Teilinterpretation von $J$. 
	
	Zu zeigen: Für alle $\beta : \X \mapsto D_I$ und alle existentiellen Formeln $A$ gilt:
	
	$$ \Ib \models A \Rightarrow \Jb \models A $$
	
	\textit{Beweis:} Induktionüber die Herleitungslänge
	
	\begin{enumerate}[(i)]
	
		\item Allgemeingültig
		\item 
		
			\begin{align*}
				\Ib \models A \land B & \Rightarrow \Ib \models A \text{ und } \Ib \models B & \text{(Def)} \\
				 & \Rightarrow \Jb \models A \text{ und } \Jb \models B & \text{(IV)} \\
				 & \Rightarrow \Jb \models A\land B & \text{(A4)}
			\end{align*}
			
			
		\item Analog zu (ii)
		
		\item Gelte $\Ib \models \ex{x} A$. Es folgt, dass für mindestens ein $d\in D_I : Ib\repl{x}{d}\models A$ gilt. Wir wissen, dass $d \in D_J$ gilt, da $D_I \subseteq D_J$ ist. Daraus folgt, dass $\Jb \models \ex{x}A$ auch gilt.
	
	
	\qed
	\end{enumerate}
\end{itemize}

\task{2}{9}

Definiere:
$$	A \equiv p \land q \lor q \land r \hspace{.4in}
	B \equiv p \lor q \land r \rightarrow p \land q \lor q \land r \hspace{.4in}
	C \equiv \lnot ( p \lor q \land r \rightarrow (\lnot \lnot p \leftrightarrow p)
$$

\begin{enumerate}
	\item Wahrheitstabelle:

	\begin{center}
	\begin{tabular}{|c|c|c||c|c|c|c|c|c|c|c|}
p & q & r & q $\land$ r & p $\lor$ q $\land$ r & p $\land$ q & A & B & $\lnot\lnot$p $\leftrightarrow$ p & C\\
\hline
tt	&	tt	&	tt	&	tt	&	tt	&	tt	&	tt		&	tt	&	tt	&	f\/f	\\
tt	&	tt	&	f\/f &	f\/f	&	tt	&	f\/f	&	f\/f		&	tt	&	tt	&	f\/f	\\
tt	&	f\/f	&	tt	&	f\/f	&	tt	&	f\/f	&	f\/f		&	f\/f	&	tt	&	f\/f	\\
tt & f\/f	&	f\/f	&	f\/f	&	tt	&	f\/f	&	f\/f		&	f\/f	&	tt	&	f\/f	\\
f\/f	& tt	&	tt	&	tt	&	tt	&	f\/f	&	tt	&	tt	&	tt	& 	f\/f	\\
f\/f	& tt	&	f\/f	&	f\/f	&	f\/f	&	f\/f	&	f\/f	&	tt	&	tt	&	f\/f	\\
f\/f	& f\/f	&	tt	&	f\/f	&	f\/f	&	f\/f	&	f\/f		&	tt	&	tt	&	f\/f	\\
f\/f	& f\/f	&	f\/f	&	f\/f	&	f\/f	&	f\/f	&	f\/f	&	tt	&	tt	&	f\/f	\\
\end{tabular}
\end{center}

	\item DNF und KNF für die Formel aus 1.(b)

\begin{align*}
\DNF{B} \equiv & \DNF{p \lor q \land r \rightarrow p \land q \lor q \land r} \\
\equiv & (p \land q \land r) \lor (p \land q \land \lnot r) \lor (\lnot p \land q\land r) \lor (\lnot p \land q \land \lnot r) \lor (\lnot p \land \lnot q \land r) \lor (\lnot p \land \lnot q \land \lnot r)
\\
\\
\KNF{B} \equiv & \KNF{p \lor q \land r \rightarrow p \land q \lor q \land r} \\
\equiv & \, (p \lor \lnot q \lor r) \land (p \lor \lnot q \lor \lnot r) \\
\equiv & \, p \lor \lnot q
\end{align*}
\end{enumerate}


\task{3}{11}

\begin{enumerate}

	\item 
	
	\begin{align*}
		\KNF{A} \equiv & \KNF{(p\to(q\to r))\to((p\to q)\to(q\to r))} & \text{(Def. A)}\\
				\equiv & \KNF{(\lnot p\lor(q\to r))\to((\lnot p \lor q) \to (\lnot q \lor r))}  & \text{(Meta)}\\
				\equiv & \KNF{(\lnot p\lor(\lnot q \lor r))\to (\lnot(\lnot p \lor q) \lor (\lnot q \lor r))}  & \text{(Meta)}\\
				\equiv & \KNF{\lnot(\lnot p\lor(\lnot q \lor r))\lor((p \land \lnot  q) \lor (\lnot q \lor r))}  & \text{(Meta)}\\
				\equiv & \KNF{\lnot(\lnot p\lor(\lnot q \lor r))\lor(((p \land \lnot  q) \lor (\lnot q \lor r)))}  & \text{(Meta)}\\
				\equiv & \KNF{(p \land (q \land \lnot r)) \lor ((p \land \lnot  q) \lor (\lnot q \lor r))}  & \text{(Meta)}\\
				\equiv & \KNF{(p \land (q \land \lnot r)) \lor ((p \land \lnot  q) \lor (\lnot q \lor r))}  & \text{(Meta)}\\
				\equiv & \KNF{p \lor ((p \land \lnot  q) \lor (\lnot q \lor r))} \land \\ 
					   & \KNF{(q \land \lnot r) \lor ((p \land \lnot  q) \lor (\lnot q \lor r))}& \text{(Hinweis)}\\
				\equiv & \KNF{(p \land \lnot  q) \lor (p \lor (\lnot q \lor r))} \land \\ 
					   & \KNF{(q \land \lnot r) \lor ((p \land \lnot  q) \lor (\lnot q \lor r))}& \text{(Klammerung)}\\
				\equiv & \KNF{p \lor (p \lor (\lnot q \lor r))} \land \KNF{\lnot q \lor (p \lor (\lnot q \lor r))} \land \\ 
				       & \KNF{q \lor ((p \land \lnot  q) \lor (\lnot q \lor r))} \land \KNF{\lnot r \lor ((p \land \lnot  q) \lor (\lnot q \lor r))}& \text{(Hinweis)}\\
				\equiv & \KNF{p \lor (\lnot q \lor r)} \land \KNF{p \lor (\lnot q \lor r)} \land \\ 
				       & \KNF{q \lor ((p \land \lnot  q) \lor (\lnot q \lor r))} \land \KNF{\lnot r \lor ((p \land \lnot  q) \lor (\lnot q \lor r))}& (A\land A \equiv A)\\
				\equiv & \KNF{p \lor (\lnot q \lor r)} \land \KNF{q \lor ((p \land \lnot  q) \lor (\lnot q \lor r))} \land \\
					   & \KNF{\lnot r \lor ((p \land \lnot  q) \lor (\lnot q \lor r))}& (A\land A \equiv A)\\
				\vdots \, &  \\
				\equiv & \,\lnot q \lor \lnot r & \text{(Draufschauen)} \\
	\end{align*}

\end{enumerate}



\pointsttl{25}
\smiley

\end{document}