\documentclass{article}
\usepackage{microtype}
\usepackage[utf8]{inputenc} 
\usepackage[a4paper, total={6in, 9.6in}]{geometry}
\usepackage{MnSymbol}
\usepackage{stmaryrd}
\usepackage{enumerate}
\usepackage{amsmath}
\usepackage{fancyhdr}
\usepackage{xcolor}
\usepackage{mathtools}

%% headers
\pagestyle{fancy}
\fancyhf{}
\rhead{Logik Für Informatiker WS19/20}
\lhead{Daniel Schubert, Anton Lydike}
\rfoot{Seite \thepage}

% simple command to display Aufgabe <num>)       ___ / <num>p.
\newcommand\task[2]{\section*{Aufgabe #1)\hfill \underline{\,\,\,\,\,\,}\,\,/#2p.}}

% Interpretation (I)
\newcommand\I{I}
% Interpretation und belegung (I, \beta)
\newcommand\Ib{\I, \beta}

%% models
\newcommand\lmodels{\leftmodels} 			% =|
\newcommand\bimodels{\leftmodels\models}	% =||=


%% table for total points
\newcommand\pointsttl[1]{\section*{Gesamtpunkte: \hfill \underline{\,\,\,\,\,\,}\,\,/#1p.}}

%% Funktionen und Prädikate
% Funktionen (arg ist anzahl der stellen)
\newcommand\func[1]{\mathcal{F}^{#1}}
% Prädikate (arg ist anzahl der stellen)
\newcommand\praed[1]{\mathcal{P}^{#1}}

%% Regeln
\newcommand\defrule[2]{\frac{#1}{#2}}

%% Funktionszahl
\newcommand\funcnum[1]{\#_{F}\, #1}

% Für ersetzungen in belegungen wie { x \mapsto d }
\newcommand\repl[2]{\{#1 \mapsto #2\}}

% für alle x .
\newcommand\fall[1]{\forall #1 \, . \,}
\newcommand\ex[1]{\exists #1 \, . \,}

% short biimplication
\newcommand\biimpl{\Leftrightarrow}

% draw a box on the right side of the page
\newcommand\qed{ \hfill $\Box$ }

% red, green, blue text:
\newcommand\red[1]{\textcolor{red}{#1}}
\newcommand\green[1]{\textcolor{green}{#1}}
\newcommand\blue[1]{\textcolor{blue}{#1}}

% more symbols: https://oeis.org/wiki/List_of_LaTeX_mathematical_symbols

\newcommand\cfgtitle[1]{\title{\vspace{-1.5cm}Übungsblatt #1\\%
\begin{large} Übungsgruppe 1 \end{large}} \lfoot{Übungsblatt #1}\cfoot{Übungsgruppe 1}}
\author{Daniel Schubert\\Anton Lydike}


\newcommand{\FV}{\text{FV}}
\newcommand{\BV}{\text{BV}}

% Logik f.I.

\cfgtitle{7}
\date{Donnerstag 5.12.2019}

\newcommand\X{\mathcal{X}}


\begin{document}
\maketitle
\thispagestyle{fancy}

\task{1}{9}

\begin{enumerate}[1)]
	\item \begin{align*}
		\FV(A) & = \FV (\fall{v} (\ex{x} P(x) \land q \to \ex{z} P(x)) \land Q(v, w, x)) \\
		  & = \FV((\ex{x} P(x) \land q \to \ex{z} P(x)) \land Q(v, w, x)) \setminus \{v\} & \text{A5} \\
		  & = (\FV(\ex{x} P(x) \land q \to \ex{z} P(x)) \cup \FV ( Q(v, w, x)) ) \setminus \{v\}  & \text{A4} \\
		  & = ((\FV(P(x) \land q \to \ex{z} P(x)) \setminus \{x\}) \cup \FV ( Q(v, w, x)) ) \setminus \{v\} & \text{A5} \\
		  & = (((\FV(P(x)) \cup \FV(q) \cup \FV(\ex{z} P(x))) \setminus \{x\}) \cup \FV(v) \cup \FV(w) \cup \FV(x)) \setminus \{v\}  & \text{A4} \\
		  & = (((\FV(x) \cup \{q\} \cup \FV(\ex{z} P(x))) \setminus \{x\}) \cup \{v\} \cup \{w\} \cup \{x\}) \setminus \{v\} & \text{A4} \\
		  & = (((\{x\} \cup \{q\} \cup (\FV(P(x)) \setminus \{z\} )) \setminus \{x\}) \cup \{v,w,x\}) \setminus \{v\} & \text{A5} \\
		  & = (((\{x, q\} \cup (\{x\} \setminus \{z\})) \setminus \{x\}) \cup \{v,w,x\}) \setminus \{v\} & \text{A4} \\
		  & = (((\{x, q\} \cup \{x\}) \setminus \{x\}) \cup \{v,w,x\}) \setminus \{v\} & \text{Mengenlehre} \\
		  & = ((\{x, q\} \setminus \{x\}) \cup \{v,w,x\}) \setminus \{v\} \\
		  & = (\{q\} \cup \{v,w,x\}) \setminus \{v\} \\
		  & = \{q, v,w,x\} \setminus \{v\} \\
		  & = \{q,w,x\}\\ \\
	\BV(B) & = \BV (\fall{v} (\ex{x} P(x) \land q \to \ex{z} P(x)) \land Q(v, w, x)) \\
		  & = \BV((\ex{x} P(x) \land q \to \ex{z} P(x)) \land Q(v, w, x)) \cup \{v\} & \text{A5} \\
		  & = (\BV(\ex{x} P(x) \land q \to \ex{z} P(x)) \cup \BV ( Q(v, w, x)) ) \cup \{v\}  & \text{A4} \\
		  & = ((\BV(P(x) \land q \to \ex{z} P(x)) \cup \{x\}) \cup \BV ( Q(v, w, x)) ) \cup \{v\} & \text{A5} \\
		  & = (((\BV(P(x)) \cup \BV(q) \cup \BV(\ex{z} P(x))) \cup \{x\}) \cup \BV(v) \cup \BV(w) \cup \BV(x)) \cup \{v\}  & \text{A4} \\
		  & = (((\BV(x) \cup \emptyset \cup \BV(\ex{z} P(x))) \cup \{x\}) \cup \emptyset \cup \emptyset \cup \emptyset) \cup \{v\} & \text{A4} \\
		  & = ((\emptyset \cup \emptyset \cup (\BV(P(x)) \cup \{z\} )) \cup \{x\}) \cup \{v\} & \text{A5} \\
		  & = ((\emptyset \cup \{z\} ) \cup \{x\}) \cup \{v\} & \text{A5} \\
		  & = \{z, x, v\} & \text{Mengenlehre} \\
	\end{align*}
	
% & = \FV(\fall{x_1} \fall{y_1} R(x_1, y_1) \land T_1(x_1, z, x_2) \land x_2 \ne y_2\to \ex{y_2} T_2(y_1, z, y_2) \land R(x_2, y_2)) \\
	\item \begin{align*}
		\FV(B) 
		       & = \{x_2, y_2\} \\ \\
		\BV(B) & = \{x_1, y_1, y_2\}
	\end{align*}
\end{enumerate}

\task{2}{7}

\begin{enumerate}[a)]
	\item \begin{align*}
		& (\fall{x_0}f(x_1,x_2) = g(x_2)) \, \repl{x_2}{f(x_1,x_2)} \\
 \equiv & \fall{x_0}(f(x_1,x_2) = g(x_2)) \, \repl{x_2}{f(x_1,x_2)} \\
 \equiv & \fall{x_0}(f(x_1,x_2)\, \repl{x_2}{f(x_1,x_2)} = g(x_2)\, \repl{x_2}{f(x_1,x_2)})  \\
 \equiv & \fall{x_0}(f(x_1\, \repl{x_2}{f(x_1,x_2)}),x_2\, \repl{x_2}{f(x_1,x_2)})) = g(x_2\, \repl{x_2}{f(x_1,x_2)}))  \\
 \equiv & \fall{x_0}(f(x_1,f(x_1,x_2)) = g(f(x_1,x_2)))  \\
	\end{align*}
	\item \begin{align*}
		& (\fall{x_0} \ex{x_1}f(x_1, x_2) = g(x_2) \to \ex{x_2} P(x_1, g(x_2))) \repl{x_2}{f(x_1, x_2)} \\
		\equiv & \fall{x_0} \ex{x_3}(f(x_3, x_2) = g(x_2) \to \ex{x_2} P(x_1, g(x_2))) \repl{x_2}{f(x_1, x_2)} \\
		\equiv & \fall{x_0} \ex{x_3}(f(x_3, x_2) = g(x_2))\repl{x_2}{f(x_1, x_2)} \to (\ex{x_2} P(x_1, g(x_2))) \repl{x_2}{f(x_1, x_2)} \\
		\equiv & \fall{x_0} \ex{x_3}f(x_3, x_2)\repl{x_2}{f(x_1, x_2)} = g(x_2)\repl{x_2}{f(x_1, x_2)} \to \ex{x_2} P(x_1, g(x_2)) \\
		\equiv & \fall{x_0} \ex{x_3}f(x_3, x_2)\repl{x_2}{f(x_1, x_2)} = g(x_2)\repl{x_2}{f(x_1, x_2)} \to \ex{x_2} P(x_1, g(x_2)) \\
		\equiv & \fall{x_0} \ex{x_3}f(x_3, f(x_1, x_2)) = g(f(x_1, x_2)) \to \ex{x_2} P(x_1, g(x_2)) 
	\end{align*}
\end{enumerate}

\task{3}{9}

$$ A \equiv (\lnot p\lor \lnot r\lor s)\land \lnot s\land (\lnot p\lor \lnot q\lor r)\land (\lnot p\lor q\lor s) $$

\begin{enumerate}[1)]
	\item $$ F = \{\{\lnot p,\lnot r, s\},\{\lnot s\},\{\lnot p,\lnot q, r\},\{\lnot p, q, s\}\} = Res^0(F)$$
	\item Das verfahren terminiert, wenn ein $N$ existiert, so dass $\forall i > N : Res^i(F) = Res^{i+1}(F)$ gilt. Da aber für alle $i$ gilt, dass jede Menge $M \in Res^i(F)$ aus freien Variablen (oder deren Negation) von $F$ besteht, und $F$ nur endlich viele (sagen wir $n$) Freie variablen enthält, gilt $\forall i : |Res^i(F)| \leq 2^{2^{(2n)}}$. Außerdem gilt trivialerweise $\forall i : Res^i(F) \subseteq Res^{i+1}(F)$. Daraus folgt offensichtlich, dass ein $N$ existieren muss, so dass $\forall i > N : Res^i(F) = Res^{i+1}(F)$ gilt. Damit terminiert das Verfahren immer.
	\item \begin{align*}
		Res^1(F) & = Res^0(F) \cup \{R \,|\, R \text{ Resolvente von } Res^0(F)\} \\
		         & = Res^0(F) \cup \{\{\lnot p, \lnot r\},\{\lnot p, q\},\{\lnot r, s\},\{ \lnot q, r\},\{ q,s\}\} \\ \\
		Res^2(F) & = Res^1(F) \cup \{R \,|\, R \text{ Resolvente von } Res^1(F)\} \\
		         & = Res^1(F) \cup \{\{\lnot p, \lnot r \},\{ \lnot p, q \},\{ \lnot r \},\{ \lnot q \},\{ \lnot r, s \},\{ \lnot q, r \},\{ q,s \},\{ q \},\{ \lnot p, \lnot q, s \},\\ &\{ \lnot p, r, s \},\{ \lnot p, \lnot q \},\{ \lnot p, r \},\{ \lnot p, \lnot q, s \},\{ \lnot p, r, s \},\{ \lnot r, q \},\{ \lnot q, s \},\{ r,s \}\}\\
	\end{align*}
	
	Da $\{\lnot r\}, \{r, s\} \in Res^2(F)$ gilt, folgt $\{s\} \in Res^3(F)$. Da $\{\lnot s\} \in Res^0(F)$ ist, gilt $\emptyset \in Res^4(F)$.
	\item  Laut Satz 2.7 gilt: $\emptyset \in Res^*(F) \land \{p\} \in Res^*(F) \Rightarrow \{\lnot p\} \in Res^*(F)$. In unserem Fall folgt, dass $\{\lnot p\} \in Res^*(F)$.
\end{enumerate}
\pointsttl{25}

\vfill\centering\includesvg[scale=.5]{../../smile.svg}
\end{document}