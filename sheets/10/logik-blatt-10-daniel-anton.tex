\documentclass{article}
\usepackage{microtype}
\usepackage[utf8]{inputenc} 
\usepackage[a4paper, total={6in, 9.6in}]{geometry}
\usepackage{MnSymbol}
\usepackage{stmaryrd}
\usepackage{enumerate}
\usepackage{amsmath}
\usepackage{fancyhdr}
\usepackage{xcolor}
\usepackage{mathtools}

%% headers
\pagestyle{fancy}
\fancyhf{}
\rhead{Logik Für Informatiker WS19/20}
\lhead{Daniel Schubert, Anton Lydike}
\rfoot{Seite \thepage}

% simple command to display Aufgabe <num>)       ___ / <num>p.
\newcommand\task[2]{\section*{Aufgabe #1)\hfill \underline{\,\,\,\,\,\,}\,\,/#2p.}}

% Interpretation (I)
\newcommand\I{I}
% Interpretation und belegung (I, \beta)
\newcommand\Ib{\I, \beta}

%% models
\newcommand\lmodels{\leftmodels} 			% =|
\newcommand\bimodels{\leftmodels\models}	% =||=


%% table for total points
\newcommand\pointsttl[1]{\section*{Gesamtpunkte: \hfill \underline{\,\,\,\,\,\,}\,\,/#1p.}}

%% Funktionen und Prädikate
% Funktionen (arg ist anzahl der stellen)
\newcommand\func[1]{\mathcal{F}^{#1}}
% Prädikate (arg ist anzahl der stellen)
\newcommand\praed[1]{\mathcal{P}^{#1}}

%% Regeln
\newcommand\defrule[2]{\frac{#1}{#2}}

%% Funktionszahl
\newcommand\funcnum[1]{\#_{F}\, #1}

% Für ersetzungen in belegungen wie { x \mapsto d }
\newcommand\repl[2]{\{#1 \mapsto #2\}}

% für alle x .
\newcommand\fall[1]{\forall #1 \, . \,}
\newcommand\ex[1]{\exists #1 \, . \,}

% short biimplication
\newcommand\biimpl{\Leftrightarrow}

% draw a box on the right side of the page
\newcommand\qed{ \hfill $\Box$ }

% red, green, blue text:
\newcommand\red[1]{\textcolor{red}{#1}}
\newcommand\green[1]{\textcolor{green}{#1}}
\newcommand\blue[1]{\textcolor{blue}{#1}}

% more symbols: https://oeis.org/wiki/List_of_LaTeX_mathematical_symbols

\newcommand\cfgtitle[1]{\title{\vspace{-1.5cm}Übungsblatt #1\\%
\begin{large} Übungsgruppe 1 \end{large}} \lfoot{Übungsblatt #1}\cfoot{Übungsgruppe 1}}
\author{Daniel Schubert\\Anton Lydike}

% Logik f.I.

\cfgtitle{10}
\date{Donnerstag 9.1.2020}

\begin{document}
\maketitle
\thispagestyle{fancy}

\task{1}{5}
\begin{align}
	\{21=21\}\, y &= 21\,  \{y=21\} & & & (=_p) \\
	\{x=12\}\, y &= 21\,  \{y=21\} & x=12 \Rightarrow 21 = 21 && (\text{K})&(1) \\
	\{2\cdot y = 42\}\, x &= 2\cdot y\,  \{x=42\} & & & (=_p) \\
	\{y=21\}\, x &= 2\cdot y\,  \{x=42\} & 2\cdot y = 42 \Rightarrow y = 21 && (\text{K})&(3) \\
	\{x=12\}\, y &= 21;\, x = 2\cdot y\,  \{x=42\} & & & (\text{Seq})&(2)(4) 
\end{align}
\task{2}{4}

\begin{align*}
	\text{A} &\equiv x = 0 \land y = 0 \land i = 0 \land z = 1 \land n = n_0 \land n > 0 \\
	\text{B} &\equiv x = n_0^3\\
\end{align*}
$$ \{\text{A}\}\,\,\, \text{while}(i < n) \{i=i+1;\,x=x+y+z;\,y=y+2 \cdot z + 1;\,z=z+3;\}\,\,\, \{\text{B}\} $$

Dabei wird die Schwache Semantik unterstellt, da keine Terminierungsgröße 
angegeben wird. Eine gute terminierungsgröße wäre $z=-i$, die nach jeder ausführen 
des Schleifenkörpers kleiner wird, aber nach unten durch $-(n+1)$ beschränkt ist.
\task{3}{9}
\begin{enumerate}[a)]
	\item \begin{itemize}
		\item P$_1(x,z) := \ex{y} \text{P}(x,y,z)$
		\item P$_i(x,z) := \ex{z_1}\ex{y_1} \text{P}(x,y,z_1) \land \text{P}_{i-1}(z_1,z)$
	\end{itemize}
	\item \begin{itemize}
		\item Alt$^\text{P}_1(x,z) := \ex{y} \text{P}(x,y,z)$
		\item Alt$^\text{P}_i(x,z) := \ex{z_1}\ex{y_1} \text{P}(x,y,z_1) \land \text{Alt}^\text{R}_{i-1}(z_1,z)$
		\item Alt$^\text{R}_1(x,z) := \ex{y} \text{R}(x,y,z)$
		\item Alt$^\text{R}_i(x,z) := \ex{z_1}\ex{y_1} \text{R}(x,y,z_1) \land \text{Alt}^\text{P}_{i-1}(z_1,z)$
		\item Alt$_i(x,z) := \text{Alt}^\text{P}_i(x,y) \lor \text{Alt}^\text{R}_i(x,y) $
	\end{itemize}
\end{enumerate}
\task{4}{7}

\pointsttl{25}

\vfill\centering\includesvg[scale=.5]{../../smile.svg}
\end{document}
