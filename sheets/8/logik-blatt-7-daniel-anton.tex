\documentclass[leqno]{article}
\usepackage{microtype}
\usepackage[utf8]{inputenc}
\usepackage[a4paper, total={7in, 9.6in}]{geometry}
\usepackage{enumerate}
\usepackage{amsmath}
\usepackage{amssymb}
\usepackage{mathabx}
\usepackage{fancyhdr}
\usepackage{xcolor}
\usepackage{mathtools}
\usepackage{graphicx}
\usepackage{svg}


%% headers
\pagestyle{fancy}
\fancyhf{}
\rhead{Logik Für Informatiker WS19/20}
\lhead{Daniel Schubert, Anton Lydike}
\rfoot{Seite \thepage}

\usepackage{etoolbox}
\AtBeginEnvironment{align}{\setcounter{equation}{0}}

% simple command to display Aufgabe <num>)       ___ / <num>p.
\newcommand\task[2]{\section*{Aufgabe #1)\hfill \underline{\,\,\,\,\,\,}\,\,/#2p.}}

% Interpretation (I)
\newcommand\I{I}
% Interpretation und belegung (I, \beta)
\newcommand\Ib{\I, \beta}
% Teilinterpretation (J)
\newcommand\J{J}
% Teilinterpretation und belegung (J, \beta)
\newcommand\Jb{\J, \beta}

%% models
%\newcommand{\models}{\rightmodels}
\renewcommand{\models}{\vDash}
\newcommand\lmodels{\Dashv} 			% =|
\newcommand\bimodels{\lmodels\models}	% =||=


\newcommand\zuzeigen{\mathrm{Z\kern-.3em\raise-0.5ex\hbox{Z}}}

%% table for total points
\newcommand\pointsttl[1]{\section*{Gesamtpunkte: \hfill \underline{\,\,\,\,\,\,}\,\,/#1p.}}

%% Funktionen und Prädikate
% Funktionen (arg ist anzahl der stellen)
\newcommand\func[1]{\mathcal{F}^{#1}}
% Prädikate (arg ist anzahl der stellen)
\newcommand\praed[1]{\mathcal{P}^{#1}}

%% Regeln
\newcommand\defrule[2]{\frac{#1}{#2}}

%% Funktionszahl
\newcommand\funcnum[1]{\#_{F}\, #1}

% Für ersetzungen in belegungen wie { x \mapsto d }
\newcommand\repl[2]{\{#1 \mapsto #2\}}

% für alle x .
\newcommand\fall[1]{\forall #1 \, . \,}
\newcommand\ex[1]{\exists #1 \, . \,}

% short biimplication
\newcommand\biimpl{\Leftrightarrow}

% draw a box on the right side of the page
\newcommand\qed{ \hfill $\Box$ }

% red, green, blue text:
\newcommand\red[1]{\textcolor{red}{#1}}
\newcommand\green[1]{\textcolor{green}{#1}}
\newcommand\blue[1]{\textcolor{blue}{#1}}

% more symbols: https://oeis.org/wiki/List_of_LaTeX_mathematical_symbols

\newcommand\cfgtitle[1]{\title{\vspace{-1.5cm}Übungsblatt #1\\%
\begin{large} Übungsgruppe 1 \end{large}} \lfoot{Übungsblatt #1}\cfoot{Übungsgruppe 1}}
\author{Daniel Schubert\\Anton Lydike}

\newcommand\smiley{\vfill\centering\includesvg[scale=.5]{../../smile.svg}}


\newcommand\N{\mathbb{N}}
\newcommand\Z{\mathbb{Z}}

\newcommand\ff{f\!\! f}


\renewcommand{\iff}{\Leftrightarrow}
\newcommand{\FV}{\text{FV}}
\newcommand{\BV}{\text{BV}}

\DeclareMathOperator{\anzahl}{anzahl}

% Logik f.I.

\cfgtitle{8}
\date{Donnerstag 12.12.2019}

\newcommand{\vdashg}{\vdash_\text{G}}

\begin{document}
\maketitle
\thispagestyle{fancy}

\task{1}{8}

\begin{enumerate}
	\item \begin{align}
		A, \lnot \lnot B & \vdashg A & \text{A}\\
		A, \lnot B & \vdashg \lnot B & \text{A} \\
		A, \lnot A & \vdashg \lnot B & \text{\lnot L} & \text{(1)} \\
		A, (\lnot A \lor \lnot B) & \vdashg \lnot B & \text{\lor L} & \text{(2)(3)} \\
		A,B &\vdashg \lnot(\lnot A \lor \lnot B)  & \text{\lnot R}\\
		(A \land B) &\vdashg \lnot(\lnot A \lor \lnot B)  & \text{\land L}\\
		&\vdashg (A \land B) \to \lnot(\lnot A \lor \lnot B)  & \text{\to R}
	\end{align}
	\item \begin{align}
		A,B,C,\lnot C & \vdashg A & \text{A} \\
		A,\lnot C & \vdashg A  & \text{A} \\
		A,B\land C, \lnot C & \vdashg A  & \text{A} \\
		B,C,A & \vdashg \lnot(\lnot C)  & \text{A} \\
		B,\lnot C, C & \vdashg A & \text{\lnot R} \\
		B\land C, B,\lnot C & \vdashg A & \text{\land L} \\
		A,B,\lnot C & \vdashg A & \text{} \\
		A \lor (B\land C), A, \lnot C &\vdashg A& \text{\lor L} \\
		A \lor (B \land C), B, \lnot C &\vdashg A & \text{\lor L} \\
		A \lor (B \land C), A\lor B, \lnot C & \vdashg A & \text{\lor L} \\
		A \lor (B\land C), A\lor B & \vdashg A\lor C& \text{\lor R} \\
		A \lor (B\land C), A\lor B & \vdashg A\lor B& \text{\lor R} \\
		A \lor (B\land C), A\lor B & \vdashg (A\lor B) \land (A\lor C)& \text{\land R} \\
		& \vdashg (A\lor (B \land C)) \to ((A\lor B) \land (A\lor C)) & \text{\to R} \\
	\end{align}
\end{enumerate}


\task{2}{8}

\begin{enumerate}
	\item  \begin{align}
		\vdash & y=x \to (x=y \to y=x) & \text{Ax2} \\
		\vdash & \fall{x} \fall{y} (y=x \to (x=y \to y=x)) & \text{ge} \\
		\vdash & (\fall{x} \fall{y} (y=x \to (x=y \to y=x))) \to ((\fall{x} \fall{y} y=x) \to (\fall{x} \fall{y} x=y \to y=x)) & \text{D\forall} \\
		\vdash & (\fall{x} \fall{y} y=x) \to (\fall{x} \fall{y} x=y \to y=x) & \text{MP} \\
		\vdash & \fall{x} \fall{y} x=y \to y=x & \text{MP}
	\end{align}
	\item \begin{align}
		& \{P(y)\to \fall{x} Q(x)\} \vdash P(y) \to \fall{x} Q(x) & \text{Trivial} \\
		& \{P(y)\to \fall{x} Q(x)\} \vdash \fall{x} P(y) \to \fall{x} Q(x) & \text{ge(1)} \\
		& \{P(y)\to \fall{x} Q(x)\} \vdash \fall{x} P(y) \to Q(x) & \text{D\forall - Umgekehrt}
	\end{align}
\end{enumerate}


\task{3}{9}

\begin{enumerate}
	\item \begin{enumerate}
		\item  $\fall{e} K(e) \to \fall{z} K(z) \land \ex{x}. P(x,e,z)$
		\item $\anzahl \in P^2$, $\anzahl^{I} := \# (e \overset{x}{\to} z)$ (zu deutsch: Anzahl der Kanten x von e zu z.)\\
		 $\fall{e} K(e) \to \fall{x} P(x,e,z) \to \anzahl(P)=1$
		\item  $\fall{k} K(k) \to \fall{i} K(i) \to \fall{x} P(x,i,i) \land (\lnot \ex{z} P(z,i,k) \land i \neq K)$
	\end{enumerate}
	\item \begin{enumerate}
		\item ZP,ZR $\in$ $P^1$\\
ZP$^{I} := x$ ist Zustand aus $G_P$\\
ZR$^{I} := x$ ist Zustand aus $G_R$ \\
		\item $\fall{x_1} \fall{x_2} \fall{y_1} \fall{y_2} \to ZP(x_1) \land ZP(x_2) \land ZR(y_1) \land ZR(y_2) \to x_1 = y_1 \to ZP(x_1)=x_2 \land ZR(y_1)=y_2 \to x_2 = y_2$
	\end{enumerate}
\end{enumerate}

\pointsttl{25}

\vfill\centering\includesvg[scale=.5]{../../smile.svg}
\end{document}