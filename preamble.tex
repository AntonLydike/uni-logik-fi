\documentclass[leqno]{article}
\usepackage{microtype}
\usepackage[utf8]{inputenc}
\usepackage[a4paper, total={7in, 9.6in}]{geometry}
\usepackage{enumerate}
\usepackage{amsmath}
\usepackage{amssymb}
\usepackage{mathabx}
\usepackage{fancyhdr}
\usepackage{xcolor}
\usepackage{mathtools}
\usepackage{graphicx}
\usepackage{svg}


%% headers
\pagestyle{fancy}
\fancyhf{}
\rhead{Logik Für Informatiker WS19/20}
\lhead{Daniel Schubert, Anton Lydike}
\rfoot{Seite \thepage}

\usepackage{etoolbox}
\AtBeginEnvironment{align}{\setcounter{equation}{0}}

% simple command to display Aufgabe <num>)       ___ / <num>p.
\newcommand\task[2]{\section*{Aufgabe #1)\hfill \underline{\,\,\,\,\,\,}\,\,/#2p.}}

% Interpretation (I)
\newcommand\I{I}
% Interpretation und belegung (I, \beta)
\newcommand\Ib{\I, \beta}
% Teilinterpretation (J)
\newcommand\J{J}
% Teilinterpretation und belegung (J, \beta)
\newcommand\Jb{\J, \beta}

%% models
%\newcommand{\models}{\rightmodels}
\renewcommand{\models}{\vDash}
\newcommand\lmodels{\Dashv} 			% =|
\newcommand\bimodels{\lmodels\models}	% =||=


\newcommand\zuzeigen{\mathrm{Z\kern-.3em\raise-0.5ex\hbox{Z}}}

%% table for total points
\newcommand\pointsttl[1]{\section*{Gesamtpunkte: \hfill \underline{\,\,\,\,\,\,}\,\,/#1p.}}

%% Funktionen und Prädikate
% Funktionen (arg ist anzahl der stellen)
\newcommand\func[1]{\mathcal{F}^{#1}}
% Prädikate (arg ist anzahl der stellen)
\newcommand\praed[1]{\mathcal{P}^{#1}}


%% Regeln
\newcommand\defrule[2]{\frac{#1}{#2}}

%% Funktionszahl
\newcommand\funcnum[1]{\#_{F}\, #1}

% Für ersetzungen in belegungen wie { x \mapsto d }
\newcommand\repl[2]{\{#1 \mapsto #2\}}

% für alle x .
\newcommand\fall[1]{\forall #1 \, . \,}
\newcommand\ex[1]{\exists #1 \, . \,}

% short biimplication
\newcommand\biimpl{\Leftrightarrow}

% draw a box on the right side of the page
\newcommand\qed{ \hfill $\Box$ }

% red, green, blue text:
\newcommand\red[1]{\textcolor{red}{#1}}
\newcommand\green[1]{\textcolor{green}{#1}}
\newcommand\blue[1]{\textcolor{blue}{#1}}

% more symbols: https://oeis.org/wiki/List_of_LaTeX_mathematical_symbols

\newcommand\cfgtitle[1]{\title{\vspace{-1.5cm}Übungsblatt #1\\%
\begin{large} Übungsgruppe 1 \end{large}} \lfoot{Übungsblatt #1}\cfoot{Übungsgruppe 1}}
\author{Daniel Schubert\\Anton Lydike}

\newcommand\smiley{\vfill\centering\includesvg[scale=.5]{../../smile.svg}}


\newcommand\N{\mathbb{N}}
\newcommand\Z{\mathbb{Z}}

\newcommand\ff{f\!\! f}

%% temporale logik
\newcommand\G{\mathbf{G}}
\newcommand\X{\mathbf{X}}
\newcommand\F{\mathbf{F}}
\newcommand\U{\mathbf{U}}
